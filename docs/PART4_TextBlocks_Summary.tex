%%%%%%%%%%%%%%%%%%%%%%%%%%%%%%%%%%%%%%%%%%%%%%%%%%%%%%%%%%%%%%%%%%%%%%%%%%%%%%%
% PART 4: TEXT BLOCKS - Results Overview, Limitations, Recommendations
%%%%%%%%%%%%%%%%%%%%%%%%%%%%%%%%%%%%%%%%%%%%%%%%%%%%%%%%%%%%%%%%%%%%%%%%%%%%%%%

%=============================================================================
% TEXT BLOCK 1: Results Overview and Key Findings
% LOCATION: Start of Section 5.5 or as new section
% FILE: docs/doc/5.txt
% ACTION: Insert as introductory section before detailed results
%=============================================================================

\section{Results Overview and Key Findings}
\label{sec:ch5_results_overview}

The simulation experiment evaluated four opening-time scenarios (Baseline 06:00--24:00, Scenario 1 07:00--23:00, Scenario 2 08:00--22:00, Scenario 3 08:00--20:00) across three independent replications of 30-day operational periods each. Key findings are summarized below:

\paragraph{Service Level Robustness.}
FG outbound SLA compliance remains above 89\% across all scenarios, with a maximum degradation of 2 percentage points (Baseline 91.96\% vs Scenario 2 89.97\%). Regional decomposition reveals heterogeneous risk: ROW shipments maintain perfect 100\% compliance due to next-day-only deadlines, while G2 shipments (which include same-day requirements) operate at 87.6--90.0\%. This gap indicates that service risk is concentrated in the G2 region.

\paragraph{Moderate Congestion Increase.}
Average truck waiting time increases by 21\% under the most aggressive scenario (Scenario 3: 0.86 hrs vs Baseline: 0.71 hrs). However, absolute wait times remain below 1 hour on average, and maximum waits stay below 10 hours, indicating that dock congestion does not escalate to crisis levels.

\paragraph{Throughput Reduction.}
Total outbound pallet throughput decreases from 131k (Baseline) to 106k (Scenario 3), a 19\% reduction. Inbound throughput remains stable (113k--120k), suggesting that inbound operations are more resilient to opening-hour constraints than outbound flows.

\paragraph{Underutilized Capacity.}
Counterintuitively, average dock utilization \textit{decreases} as opening hours shorten (FG dock: 34.4\% $\to$ 27.9\%). This indicates that throughput reductions dominate capacity compression, and performance degradation stems from temporal misalignment rather than capacity saturation during opening hours.

%=============================================================================
% TEXT BLOCK 2: Limitations and Scope Boundaries
% LOCATION: Section 5.6 or 5.7 (Discussion/Limitations)
% FILE: docs/doc/5.txt
% ACTION: Insert as new subsection in discussion chapter
%=============================================================================

\subsection{Limitations and Scope Boundaries}
\label{subsec:ch5_limitations}

The simulation results must be interpreted within the following scope limitations:

\paragraph{Buffer Logic Incomplete.}
As documented in Section~\ref{sec:process_logic}, the trailer buffer abstraction is implemented but not actively populated by inbound/outbound processes. Consequently, midnight backlog KPIs (Figure~\ref{fig:ch5_midnight_backlog}) reflect buffer state only and should not be interpreted as full system-wide queue-length measures. Future work should integrate buffer logic to capture end-of-day inventory dynamics.

\paragraph{Small Replication Count.}
With only 3 replications per scenario, the statistical power to detect small performance differences is limited. Standard deviations are small for most KPIs, but confidence intervals are wide. Increasing replication count to 10--30 would enable more robust statistical inference.

\paragraph{No Arrival Smoothing.}
The \texttt{arrival\_smoothing} flag was set to \texttt{False} for all scenarios. Testing smoothing interventions (e.g., incentivizing off-peak arrivals) may reveal additional operational flexibility not captured in the current experiment.

\paragraph{Proportional FTE Scaling Assumption.}
The workforce module scales baseline FTE proportionally with operating hours (e.g., 12-hour window $\to$ 12/18 $\times$ baseline FTE). This linear scaling assumes no economies or diseconomies of scale in labor productivity. Empirical validation of this assumption is recommended.

\paragraph{Fixed Demand Profiles.}
Hourly arrival-rate profiles are held constant across scenarios. In practice, suppliers and customers may adjust their behavior in response to shorter DC opening hours, potentially shifting arrival patterns toward mid-day peaks. Incorporating endogenous demand responses would require stakeholder interviews or historical before-after analysis from similar operational changes.

%=============================================================================
% TEXT BLOCK 3: Operational Recommendations
% LOCATION: Section 5.7 or Chapter 6 (Conclusions/Recommendations)
% FILE: docs/doc/5.txt
% ACTION: Insert as new subsection or in conclusions chapter
%=============================================================================

\subsection{Operational Recommendations}
\label{subsec:ch5_recommendations}

Based on simulation results, the following operational adjustments are recommended to mitigate performance degradation under shorter opening hours:

\paragraph{Prioritize G2 Morning Timeslots.}
Since G2 SLA compliance drops below 90\% in Scenario 2, and morning hours (08:00--12:00) exhibit peak utilization (Figure~\ref{fig:ch5_util_hourly}), allocating priority timeslots for G2 shipments during 08:00--12:00 may reduce SLA misses. This could be implemented via reservation systems or dynamic slot allocation.

\paragraph{Shift Inbound Arrivals Earlier.}
Inbound flows show stable throughput across scenarios, suggesting resilience. However, if arrivals can be incentivized to occur earlier in the day (06:00--10:00 under Baseline, or 08:00--10:00 under Scenario 3), this would free up afternoon capacity for outbound processing, reducing end-of-day backlogs.

\paragraph{Implement Arrival Smoothing.}
The current simulation does not test arrival smoothing. Piloting demand-smoothing interventions (e.g., time-differentiated pricing, appointment scheduling) may reduce morning congestion and flatten utilization profiles, improving throughput under compressed windows.

\paragraph{Increase Morning-Shift FTE.}
Rather than scaling FTE proportionally across all hours, concentrating workforce during peak hours (08:00--14:00) may accelerate processing and reduce waiting times. This would require labor agreements permitting shift concentration.

\paragraph{Monitor G2 SLA Contractual Thresholds.}
If contractual SLA targets exceed 90\% for G2 shipments, Scenario 2 may be infeasible without operational adjustments. Stakeholder consultation is recommended to clarify acceptable service-level trade-offs.

%%%%%%%%%%%%%%%%%%%%%%%%%%%%%%%%%%%%%%%%%%%%%%%%%%%%%%%%%%%%%%%%%%%%%%%%%%%%%%%
% END OF TEXT BLOCKS
%%%%%%%%%%%%%%%%%%%%%%%%%%%%%%%%%%%%%%%%%%%%%%%%%%%%%%%%%%%%%%%%%%%%%%%%%%%%%%%

%%%%%%%%%%%%%%%%%%%%%%%%%%%%%%%%%%%%%%%%%%%%%%%%%%%%%%%%%%%%%%%%%%%%%%%%%%%%%%%
% MASTER INDEX - ALL INSERTIONS SUMMARY
%%%%%%%%%%%%%%%%%%%%%%%%%%%%%%%%%%%%%%%%%%%%%%%%%%%%%%%%%%%%%%%%%%%%%%%%%%%%%%%

% CHAPTER 4 INSERTIONS (4 items):
% - Table 4.1: FTE Configuration
% - Table 4.2: Arrival Stochasticity
% - Figure 4.X: TikZ Architecture Diagram
% - Verification itemize list

% CHAPTER 5 INSERTIONS (19 items):
% TABLES (2):
% - Table 5.1: Main Comparison Table (CRITICAL)
% - Table 5.2: Regional Breakdown

% FIGURES (13):
% - Figure 1: SLA Overall
% - Figure 1b: SLA by Region
% - Figure 2: Avg Waiting Time
% - Figure 3: Midnight Backlog (optional)
% - Figure 4: Throughput Pallets
% - Figure 4b: FG Region Pallets
% - Figure 4c: Orders vs Pallets
% - Figure 4d: FG Region Orders
% - Figure 5: Dock Utilization
% - Figure 5b: Hourly Profiles (4 subfigures)

% TEXT BLOCKS (4):
% - Results Overview (Section 5.5 intro)
% - Limitations Discussion (Section 5.6/5.7)
% - Operational Recommendations (Section 5.7 or Chapter 6)
% - Individual figure analysis paragraphs (embedded with figures)

% REQUIRED LATEX PACKAGES:
% \usepackage{tikz}
% \usepackage{subcaption}
% \usetikzlibrary{shapes,arrows,positioning}

% FIGURE PATH SETUP:
% \graphicspath{{../outputs/figures/}}

%%%%%%%%%%%%%%%%%%%%%%%%%%%%%%%%%%%%%%%%%%%%%%%%%%%%%%%%%%%%%%%%%%%%%%%%%%%%%%%
% END OF COMPLETE INSERTION GUIDE
%%%%%%%%%%%%%%%%%%%%%%%%%%%%%%%%%%%%%%%%%%%%%%%%%%%%%%%%%%%%%%%%%%%%%%%%%%%%%%%
