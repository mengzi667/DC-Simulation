%%%%%%%%%%%%%%%%%%%%%%%%%%%%%%%%%%%%%%%%%%%%%%%%%%%%%%%%%%%%%%%%%%%%%%%%%%%%%%%
% PART 3C: CHAPTER 5 FIGURES - DOCK UTILIZATION (Figures 5, 5b with 4 subfigures)
%%%%%%%%%%%%%%%%%%%%%%%%%%%%%%%%%%%%%%%%%%%%%%%%%%%%%%%%%%%%%%%%%%%%%%%%%%%%%%%

%=============================================================================
% FIGURE 5: Timeslot/Dock Utilization Overall ⭐⭐
% LOCATION: Section 5.5.4 (Dock Utilization Results)
% FILE: docs/doc/5.txt
% ACTION: Replace \label{fig:ch5_util_avg} placeholder
%=============================================================================

\begin{figure}[h]
    \centering
    \includegraphics[width=0.9\textwidth]{../outputs/figures/5_timeslot_utilization.png}
    \caption{Average dock utilization (loading and reception) by category and direction. Utilization rates decrease as opening hours shorten, indicating excess capacity relative to realized demand.}
    \label{fig:ch5_util_avg}
\end{figure}

\textbf{Analysis:} Contrary to the initial hypothesis that shorter opening hours would saturate dock capacity, Figure~\ref{fig:ch5_util_avg} shows that average utilization \textit{decreases} from Baseline to Scenario 3 (e.g., FG dock from 34.4\% to 27.9\%). This counterintuitive result indicates that throughput reductions (Figure~\ref{fig:ch5_throughput_pallets}) dominate capacity compression effects. In other words, the system processes fewer trucks per hour under shorter windows, resulting in lower utilization despite tighter time constraints.

This finding suggests that performance degradation stems primarily from \textit{temporal misalignment} (trucks arriving outside opening hours and being delayed) rather than from capacity saturation during opening hours.

%=============================================================================
% FIGURE 5b: Hourly Utilization Profiles (4 subfigures) ⭐⭐⭐
% LOCATION: Section 5.5.4, after Figure 5
% FILE: docs/doc/5.txt
% ACTION: Replace \label{fig:ch5_util_hourly} placeholder
% REQUIRES: \usepackage{subcaption}
%=============================================================================

\begin{figure}[h]
    \centering
    \begin{subfigure}[b]{0.48\textwidth}
        \includegraphics[width=\textwidth]{../outputs/figures/5b_fg__inbound__slot_utilization.png}
        \caption{FG Inbound}
        \label{fig:ch5_util_fg_in}
    \end{subfigure}
    \hfill
    \begin{subfigure}[b]{0.48\textwidth}
        \includegraphics[width=\textwidth]{../outputs/figures/5b_fg__outbound__slot_utilization.png}
        \caption{FG Outbound}
        \label{fig:ch5_util_fg_out}
    \end{subfigure}
    
    \vspace{0.3cm}
    
    \begin{subfigure}[b]{0.48\textwidth}
        \includegraphics[width=\textwidth]{../outputs/figures/5b_r&p__inbound__slot_utilization.png}
        \caption{R\&P Inbound}
        \label{fig:ch5_util_rp_in}
    \end{subfigure}
    \hfill
    \begin{subfigure}[b]{0.48\textwidth}
        \includegraphics[width=\textwidth]{../outputs/figures/5b_r&p__outbound__slot_utilization.png}
        \caption{R\&P Outbound}
        \label{fig:ch5_util_rp_out}
    \end{subfigure}
    
    \caption{Hourly dock utilization profiles by category and direction. Peak utilization occurs during morning hours (08:00--12:00) across all flows. Utilization drops to zero outside opening windows.}
    \label{fig:ch5_util_hourly}
\end{figure}

\textbf{Analysis:} Figure~\ref{fig:ch5_util_hourly} reveals distinct temporal patterns:
\begin{itemize}
    \item \textbf{Morning peak:} All flows exhibit peak utilization between 08:00--12:00, driven by early arrival patterns and same-day delivery deadlines for G2 shipments.
    \item \textbf{Afternoon taper:} Utilization declines after 14:00, particularly for outbound flows, as trucks complete processing and depart.
    \item \textbf{Hard closure:} Utilization drops to zero at closing time, confirming that no dock service occurs outside opening windows.
\end{itemize}

For Scenario 3 (08:00--20:00), the compressed window truncates the traditional 06:00--08:00 early-morning activity, forcing early arrivals to queue until opening. This temporal misalignment contributes to the increased average waiting times observed in Figure~\ref{fig:ch5_wait_mean}.

%%%%%%%%%%%%%%%%%%%%%%%%%%%%%%%%%%%%%%%%%%%%%%%%%%%%%%%%%%%%%%%%%%%%%%%%%%%%%%%
% END OF PART 3C - All 13 figures now complete
%%%%%%%%%%%%%%%%%%%%%%%%%%%%%%%%%%%%%%%%%%%%%%%%%%%%%%%%%%%%%%%%%%%%%%%%%%%%%%%

% REQUIRED PACKAGES (add to preamble):
% \usepackage{subcaption}
% \graphicspath{{../outputs/figures/}}
