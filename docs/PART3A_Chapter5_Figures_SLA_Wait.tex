%%%%%%%%%%%%%%%%%%%%%%%%%%%%%%%%%%%%%%%%%%%%%%%%%%%%%%%%%%%%%%%%%%%%%%%%%%%%%%%
% PART 3A: CHAPTER 5 FIGURES - SERVICE LEVEL & CONGESTION (Figures 1, 1b, 2, 3)
%%%%%%%%%%%%%%%%%%%%%%%%%%%%%%%%%%%%%%%%%%%%%%%%%%%%%%%%%%%%%%%%%%%%%%%%%%%%%%%

%=============================================================================
% FIGURE 1: SLA Overall Compliance ⭐⭐⭐
% LOCATION: Section 5.5.2 (Service Level Results)
% FILE: docs/doc/5.txt
% ACTION: Replace \label{fig:ch5_sla_overall} placeholder
%=============================================================================

\begin{figure}[h]
    \centering
    \includegraphics[width=0.85\textwidth]{../outputs/figures/1_sla_compliance_rate.png}
    \caption{FG outbound SLA compliance rate across scenarios. All scenarios maintain above 89\% compliance. Baseline: 91.96\%, Scenario 1: 91.39\%, Scenario 2: 89.97\%, Scenario 3: 91.06\%. Error bars represent standard deviation across 3 replications.}
    \label{fig:ch5_sla_overall}
\end{figure}

\textbf{Analysis:} Figure~\ref{fig:ch5_sla_overall} demonstrates that SLA compliance remains robust across all opening-time configurations. Scenario 2 (08:00--22:00) exhibits the lowest compliance at 89.97\%, representing a 2 percentage point reduction from baseline. However, all scenarios maintain service levels within acceptable operational ranges, indicating that shorter opening windows do not catastrophically degrade service performance under current demand and capacity assumptions.

%=============================================================================
% FIGURE 1b: SLA by Region (G2 vs ROW) ⭐⭐⭐
% LOCATION: Section 5.5.2, immediately after Figure 1
% FILE: docs/doc/5.txt
% ACTION: Replace \label{fig:ch5_sla_region} placeholder
%=============================================================================

\begin{figure}[h]
    \centering
    \includegraphics[width=0.9\textwidth]{../outputs/figures/1b_sla_by_region.png}
    \caption{FG outbound SLA compliance rate by destination region. G2 region (80\% of demand) shows 87.6--90.0\% compliance, while ROW region maintains perfect 100\% across all scenarios.}
    \label{fig:ch5_sla_region}
\end{figure}

\textbf{Analysis:} Regional decomposition (Figure~\ref{fig:ch5_sla_region}) reveals critical heterogeneity in service risk. ROW shipments achieve perfect SLA compliance (100\%) across all scenarios, as all ROW trucks are assigned next-day deadlines only. In contrast, G2 shipments, which include same-day delivery requirements, exhibit SLA rates between 87.6\% and 90.0\%. This 10--12 percentage-point gap highlights that service-level risk is concentrated in the G2 region, where tight deadlines create operational pressure. If contractual SLA targets require G2 compliance above 90\%, Scenarios 2 and potentially Scenario 1 may require operational adjustments such as prioritized morning timeslots for G2 shipments.

%=============================================================================
% FIGURE 2: Average Truck Waiting Time ⭐⭐
% LOCATION: Section 5.5.3 (Congestion and Waiting-Time Results)
% FILE: docs/doc/5.txt
% ACTION: Replace \label{fig:ch5_wait_mean} placeholder
%=============================================================================

\begin{figure}[h]
    \centering
    \includegraphics[width=0.85\textwidth]{../outputs/figures/2_avg_truck_wait_time.png}
    \caption{Average truck waiting time (business hours only) across scenarios. Scenario 3 shows 21\% increase versus baseline (0.86 vs 0.71 hours). Error bars show standard deviation.}
    \label{fig:ch5_wait_mean}
\end{figure}

\textbf{Analysis:} Figure~\ref{fig:ch5_wait_mean} demonstrates that average truck waiting time increases modestly as opening hours shorten. Scenario 3 (08:00--20:00, 12 hours) exhibits the highest average wait at 0.86 hours, a 21\% increase over baseline (0.71 hours). This indicates that dock capacity becomes more saturated when trucks must compete for service within a compressed window. However, absolute wait times remain below 1 hour on average, suggesting the system retains operational feasibility. The standard deviations remain small (0.01--0.03 hours), indicating stable performance across replications.

%=============================================================================
% FIGURE 3: Midnight Backlog (Optional)
% LOCATION: Section 5.5.X or Appendix
% FILE: docs/doc/5.txt
% ACTION: Insert in results section or appendix
% NOTE: Include caveat about incomplete buffer logic
%=============================================================================

\begin{figure}[h]
    \centering
    \includegraphics[width=0.85\textwidth]{../outputs/figures/3_midnight_backlog.png}
    \caption{Midnight backlog (pending inventory) across scenarios. Shows end-of-day pallet accumulation.}
    \label{fig:ch5_midnight_backlog}
\end{figure}

\textit{Note: As documented in Section 4.X, the midnight backlog currently records buffer occupancy rather than full system-wide queue length due to incomplete buffer logic implementation. This metric should be interpreted as a placeholder for future enhancement.}

%%%%%%%%%%%%%%%%%%%%%%%%%%%%%%%%%%%%%%%%%%%%%%%%%%%%%%%%%%%%%%%%%%%%%%%%%%%%%%%
% END OF PART 3A - Continue with throughput figures in next file
%%%%%%%%%%%%%%%%%%%%%%%%%%%%%%%%%%%%%%%%%%%%%%%%%%%%%%%%%%%%%%%%%%%%%%%%%%%%%%%

% REQUIRED: Adjust figure path if needed
% \graphicspath{{../outputs/figures/}}
