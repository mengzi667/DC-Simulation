%%%%%%%%%%%%%%%%%%%%%%%%%%%%%%%%%%%%%%%%%%%%%%%%%%%%%%%%%%%%%%%%%%%%%%%%%%%%%%%
% 🎯 MASTER INSERTION GUIDE - COMPLETE INDEX
% 报告融入完整指导 - 所有LaTeX代码的总索引
%%%%%%%%%%%%%%%%%%%%%%%%%%%%%%%%%%%%%%%%%%%%%%%%%%%%%%%%%%%%%%%%%%%%%%%%%%%%%%%
% 生成日期: 2026-01-12
% 项目: DC Operation Simulation
% 总计: 4个表格 + 13张图片 + 1个架构图 + 4个文字段落 = 22个插入点
%%%%%%%%%%%%%%%%%%%%%%%%%%%%%%%%%%%%%%%%%%%%%%%%%%%%%%%%%%%%%%%%%%%%%%%%%%%%%%%

%%%%%%%%%%%%%%%%%%%%%%%%%%%%%%%%%%%%%%%%%%%%%%%%%%%%%%%%%%%%%%%%%%%%%%%%%%%%%%%
% 📂 文件组织结构
%%%%%%%%%%%%%%%%%%%%%%%%%%%%%%%%%%%%%%%%%%%%%%%%%%%%%%%%%%%%%%%%%%%%%%%%%%%%%%%

% 本指导分为6个文件(按章节和内容类型):
% 
% 1. PART1_Chapter4_Tables_Figures.tex
%    - 第4章所有插入内容
%    - 包含: 2个表格 + 1个简化TikZ图 + 1个列表
%
% 2. ARCHITECTURE_DIAGRAMS.tex ⭐⭐⭐ 新增
%    - 完整架构图集(3个详细TikZ图)
%    - 包含: Inbound架构 + Outbound架构 + 总体设计架构
%
% 3. PART2_Chapter5_Tables.tex
%    - 第5章所有表格
%    - 包含: 主对比表 + 区域分解表
%
% 4. PART3A_Chapter5_Figures_SLA_Wait.tex
%    - 第5章服务水平和拥堵图片
%    - 包含: 图1, 1b, 2, 3
%
% 5. PART3B_Chapter5_Figures_Throughput.tex
%    - 第5章吞吐量图片
%    - 包含: 图4, 4b, 4c, 4d
%
% 6. PART3C_Chapter5_Figures_Utilization.tex
%    - 第5章利用率图片
%    - 包含: 图5, 5b (4个subfigure)
%
% 7. PART4_TextBlocks_Summary.tex
%    - 文字段落 + 总结
%    - 包含: 结果概览 + 局限性 + 建议
%
% 8. THIS FILE (MASTER_INDEX.tex)
%    - 总索引和快速导航

%%%%%%%%%%%%%%%%%%%%%%%%%%%%%%%%%%%%%%%%%%%%%%%%%%%%%%%%%%%%%%%%%%%%%%%%%%%%%%%
% 📋 快速查找表 - 按重要性排序
%%%%%%%%%%%%%%%%%%%%%%%%%%%%%%%%%%%%%%%%%%%%%%%%%%%%%%%%%%%%%%%%%%%%%%%%%%%%%%%

% ⭐⭐⭐ 必须插入 (10项 - 新增3个架构图) - 报告核心内容
% ──────────────────────────────────────────────────────
% □ 总体架构图  Entity/Process/Resource   ARCHITECTURE_DIAGRAMS, line 190-290
% □ Inbound架构 流程+约束详解             ARCHITECTURE_DIAGRAMS, line 15-65
% □ Outbound架构 两阶段+SLA               ARCHITECTURE_DIAGRAMS, line 75-140
% □ 表5.1  主对比表                       PART2, line 10-70
% □ 图1    SLA Overall                    PART3A, line 10-25
% □ 图1b   SLA by Region                  PART3A, line 30-50
% □ 图5b   小时利用率剖析(4图)             PART3C, line 40-90
%
% ⭐⭐ 强烈建议 (6项) - 补充核心发现
% ──────────────────────────────────────────────────────
% □ 图2    平均等待时间                    PART3A, line 55-70
% □ 图4    吞吐量-托盘                     PART3B, line 10-25
% □ 图4c   订单vs托盘对比                  PART3B, line 45-65
% □ 图5    码头利用率总体                  PART3C, line 10-30
% □ 表4.1  FTE配置                        PART1, line 10-25
% □ 简化架构图 (可选替代)                  PART1, line 45-95
%
% ⭐ 可选补充 (9项)
% ──────────────────────────────────────────────────────
% □ 表4.2  到达随机性                      PART1, line 30-45
% □ 表5.2  区域分解表                      PART2, line 75-95
% □ 图3    Midnight Backlog               PART3A, line 75-85
% □ 图4b   FG区域-托盘                    PART3B, line 30-40
% □ 图4d   FG区域-订单                    PART3B, line 70-85
% □ 验证列表                               PART1, line 100-110
% □ 结果概览段落                           PART4, line 10-35
% □ 局限性段落                             PART4, line 40-70
% □ 建议段落                               PART4, line 75-110

%%%%%%%%%%%%%%%%%%%%%%%%%%%%%%%%%%%%%%%%%%%%%%%%%%%%%%%%%%%%%%%%%%%%%%%%%%%%%%%
% 🎯 按章节组织的插入清单
%%%%%%%%%%%%%%%%%%%%%%%%%%%%%%%%%%%%%%%%%%%%%%%%%%%%%%%%%%%%%%%%%%%%%%%%%%%%%%%

% ═══════════════════════════════════════════════════════════════════════════
% 第4章插入点 (4个基础 + 3个新架构图 = 共7个)
% ═══════════════════════════════════════════════════════════════════════════
%
% 位置1: Section 4.4 - FTE配置小节后
% ─────────────────────────────────────
% 文件: docs/doc/4.txt, line ~380
% 内容: 表4.1 - FTE配置表
% 代码: PART1_Chapter4_Tables_Figures.tex, line 10-25
% 数据: FG(44.75 FTE, 665.43 pallet/FTE), R&P(10.025 FTE, 1308.83 pallet/FTE)
%
% 位置2: Section 4.4.2 - 到达随机性小节
% ─────────────────────────────────────
% 文件: docs/doc/4.txt, line ~200
% 内容: 表4.2 - 到达随机性表
% 代码: PART1_Chapter4_Tables_Figures.tex, line 30-45
% 替换: \label{tab:arrival_stochasticity} 占位符
%
% 位置3: Section 4.3.5 - 模型架构小节 (推荐使用新的架构图集)
% ─────────────────────────────────────
% 文件: docs/doc/4.txt, line ~150
% 
% 【选项A - 简化版】原始单图架构:
% 代码: PART1_Chapter4_Tables_Figures.tex, line 50-95
% 替换: \label{fig:ch4_architecture_overview} 占位符
% 需求: \usepackage{tikz}, \usetikzlibrary{shapes,arrows,positioning}
% 
% 【选项B - 完整版】新建3图架构集 (⭐⭐⭐ 强烈推荐):
% 代码: ARCHITECTURE_DIAGRAMS.tex
% 包含: 
%   • Inbound流程架构 (Figure 4.X-A, ~70行)
%   • Outbound流程架构 (Figure 4.X-B, ~90行)  
%   • 总体仿真设计架构 (Figure 4.X-C, ~130行)
% 需求: \usepackage{tikz}, 
%       \usetikzlibrary{shapes,arrows,positioning,fit,backgrounds}
% 
% 推荐插入位置:
%   - 总体架构图 (4.X-C) → Section 4.3 开头
%   - Inbound架构 (4.X-A) → Section 4.3.4 Inbound Process
%   - Outbound架构 (4.X-B) → Section 4.3.3 Outbound Process
%
% 位置4: Section 4.8.1 - 验证小节
% ─────────────────────────────────────
% 文件: docs/doc/4.txt, line ~620
% 内容: 验证项目列表
% 代码: PART1_Chapter4_Tables_Figures.tex, line 100-110
% 替换: /* Lines 379-381 omitted */ 注释

% ┌─────────────────────────────────────────────────────────────────────────┐
% │ 新增: 3个详细架构图 (ARCHITECTURE_DIAGRAMS.tex)                          │
% └─────────────────────────────────────────────────────────────────────────┘
%
% 位置5A: Section 4.3 - 模型架构总览 (章节开头)
% ─────────────────────────────────────
% 文件: docs/doc/4.txt, Section 4.3 开头
% 内容: 总体仿真设计架构图 (Figure 4.X-C)
% 代码: ARCHITECTURE_DIAGRAMS.tex, line 190-290
% 标签: \label{fig:simulation_architecture_overview}
% 包含: Entity (Truck, Order, Buffer)
%       Process (Arrival, Inbound, Outbound)
%       Resource (FTE, Timeslot, Dock)
%       Constraint (Hourly limits, Deadlines, Operating hours)
%       Manager (Hourly, KPI, Buffer)
%
% 位置5B: Section 4.3.4 - Inbound流程模型
% ─────────────────────────────────────
% 文件: docs/doc/4.txt, Section 4.3.4
% 内容: Inbound流程架构图 (Figure 4.X-A)
% 代码: ARCHITECTURE_DIAGRAMS.tex, line 15-65
% 标签: \label{fig:inbound_architecture}
% 包含: Arrival → Timeslot Queue → Unloading → FTE Processing
%       Buffer机制 (DC关闭时)
%       24小时处理deadline
%
% 位置5C: Section 4.3.3 - Outbound流程模型
% ─────────────────────────────────────
% 文件: docs/doc/4.txt, Section 4.3.3
% 内容: Outbound流程架构图 (Figure 4.X-B)
% 代码: ARCHITECTURE_DIAGRAMS.tex, line 75-140
% 标签: \label{fig:outbound_architecture}
% 包含: Arrival (混合模式) → FTE Processing → Timeslot Queue → Loading → SLA Check
%       两阶段流程 (先处理后装车)
%       Region分类 (G2 same-day, ROW next-day)

% ═══════════════════════════════════════════════════════════════════════════
% 第5章插入点 (共18个)
% ═══════════════════════════════════════════════════════════════════════════

% ┌─────────────────────────────────────────────────────────────────────────┐
% │ 5.1 表格部分 (2个表格)                                                    │
% └─────────────────────────────────────────────────────────────────────────┘
%
% 位置5: 新建小节 - 主要性能对比
% ─────────────────────────────────────
% 文件: docs/doc/5.txt, 在Table 5.1(scenario definitions)之后
% 内容: 新建 \subsection{Primary Performance Comparison} + 表5.1
% 代码: PART2_Chapter5_Tables.tex, line 10-70
% 标签: \label{tab:ch5_comparison_summary}
% 数据: 完整对比表 - SLA, 等待, 吞吐, 利用率, 延误 (所有真实数据)
%
% 位置6: 主对比表后或单独小节
% ─────────────────────────────────────
% 文件: docs/doc/5.txt, 主对比表之后
% 内容: 表5.2 - 区域分解表
% 代码: PART2_Chapter5_Tables.tex, line 75-95
% 标签: \label{tab:ch5_regional_breakdown}
% 数据: G2/ROW 托盘和订单分解,80/20比例验证

% ┌─────────────────────────────────────────────────────────────────────────┐
% │ 5.2 服务水平与拥堵图片 (4张)                                              │
% └─────────────────────────────────────────────────────────────────────────┘
%
% 位置7: Section 5.5.2 - 服务水平结果
% ─────────────────────────────────────
% 文件: docs/doc/5.txt, Section 5.5.2
% 内容: 图1 - SLA Overall Compliance
% 代码: PART3A_Chapter5_Figures_SLA_Wait.tex, line 10-25
% 图片: outputs/figures/1_sla_compliance_rate.png
% 替换: \label{fig:ch5_sla_overall} 占位符
% 数据: Baseline 91.96%, Scenario 2最低89.97%
%
% 位置8: 紧跟图1之后
% ─────────────────────────────────────
% 文件: docs/doc/5.txt, 图1之后立即插入
% 内容: 图1b - SLA by Region
% 代码: PART3A_Chapter5_Figures_SLA_Wait.tex, line 30-50
% 图片: outputs/figures/1b_sla_by_region.png
% 替换: \label{fig:ch5_sla_region} 占位符
% 数据: G2 87-90%, ROW 100%
%
% 位置9: Section 5.5.3 - 拥堵结果
% ─────────────────────────────────────
% 文件: docs/doc/5.txt, Section 5.5.3
% 内容: 图2 - 平均等待时间
% 代码: PART3A_Chapter5_Figures_SLA_Wait.tex, line 55-70
% 图片: outputs/figures/2_avg_truck_wait_time.png
% 替换: \label{fig:ch5_wait_mean} 占位符
% 数据: Baseline 0.71hr → Scenario 3 0.86hr (+21%)
%
% 位置10: Section 5.5.X 或附录 (可选)
% ─────────────────────────────────────
% 文件: docs/doc/5.txt, 结果小节或附录
% 内容: 图3 - Midnight Backlog
% 代码: PART3A_Chapter5_Figures_SLA_Wait.tex, line 75-85
% 图片: outputs/figures/3_midnight_backlog.png
% 注意: 需说明buffer逻辑未完全实现

% ┌─────────────────────────────────────────────────────────────────────────┐
% │ 5.3 吞吐量图片 (4张)                                                      │
% └─────────────────────────────────────────────────────────────────────────┘
%
% 位置11: Section 5.5.5 - 吞吐量结果
% ─────────────────────────────────────
% 文件: docs/doc/5.txt, Section 5.5.5
% 内容: 图4 - 吞吐量(托盘)
% 代码: PART3B_Chapter5_Figures_Throughput.tex, line 10-25
% 图片: outputs/figures/4_flow_statistics.png
% 替换: \label{fig:ch5_throughput_pallets} 占位符
% 数据: Inbound稳定113-120k, Outbound降131k→106k (-19%)
%
% 位置12: 图4之后
% ─────────────────────────────────────
% 文件: docs/doc/5.txt, 图4之后
% 内容: 图4b - FG区域分解(托盘)
% 代码: PART3B_Chapter5_Figures_Throughput.tex, line 30-40
% 图片: outputs/figures/4b_fg_outbound_by_region.png
% 替换: \label{fig:ch5_throughput_region} 占位符
% 数据: G2 80%, ROW 20%
%
% 位置13: 新建小节 - 订单vs托盘分析
% ─────────────────────────────────────
% 文件: docs/doc/5.txt, Section 5.5.5内
% 内容: 新建 \subsubsection{Order Count versus Pallet Volume} + 图4c
% 代码: PART3B_Chapter5_Figures_Throughput.tex, line 45-65
% 图片: outputs/figures/4c_flow_statistics_orders.png
% 标签: \label{fig:ch5_orders_pallets}
% 数据: FG~100 pallets/order, R&P~190 pallets/order
%
% 位置14: 图4c之后
% ─────────────────────────────────────
% 文件: docs/doc/5.txt, 图4c之后立即插入
% 内容: 图4d - FG订单区域分解
% 代码: PART3B_Chapter5_Figures_Throughput.tex, line 70-85
% 图片: outputs/figures/4d_fg_outbound_orders_by_region.png
% 标签: \label{fig:ch5_outbound_orders_region}
% 数据: G2 483-578 orders, ROW 119-144 orders

% ┌─────────────────────────────────────────────────────────────────────────┐
% │ 5.4 利用率图片 (5张 = 1主图 + 4子图)                                      │
% └─────────────────────────────────────────────────────────────────────────┘
%
% 位置15: Section 5.5.4 - 码头利用率结果
% ─────────────────────────────────────
% 文件: docs/doc/5.txt, Section 5.5.4
% 内容: 图5 - 码头利用率总体
% 代码: PART3C_Chapter5_Figures_Utilization.tex, line 10-30
% 图片: outputs/figures/5_timeslot_utilization.png
% 替换: \label{fig:ch5_util_avg} 占位符
% 数据: FG利用率 34.4%→27.9% (反直觉下降)
%
% 位置16: 图5之后
% ─────────────────────────────────────
% 文件: docs/doc/5.txt, 图5之后
% 内容: 图5b - 小时利用率剖析(2x2 subfigure布局)
% 代码: PART3C_Chapter5_Figures_Utilization.tex, line 40-90
% 图片: 4个PNG文件
%   - 5b_fg__inbound__slot_utilization.png
%   - 5b_fg__outbound__slot_utilization.png
%   - 5b_r&p__inbound__slot_utilization.png
%   - 5b_r&p__outbound__slot_utilization.png
% 替换: \label{fig:ch5_util_hourly} 占位符
% 需求: \usepackage{subcaption}
% 发现: 高峰08:00-12:00, 下午递减, 营业时间外为0

% ┌─────────────────────────────────────────────────────────────────────────┐
% │ 5.5 文字段落 (3个主要段落 + 多个图片分析)                                 │
% └─────────────────────────────────────────────────────────────────────────┘
%
% 位置17: Section 5.5 开头或新建结果概览节
% ─────────────────────────────────────
% 文件: docs/doc/5.txt, Section 5.5开头
% 内容: 新建 \section{Results Overview and Key Findings}
% 代码: PART4_TextBlocks_Summary.tex, line 10-35
% 包含: 4个段落 - Service Level, Congestion, Throughput, Capacity
%
% 位置18: Section 5.6/5.7 - 讨论章节
% ─────────────────────────────────────
% 文件: docs/doc/5.txt, 讨论章节
% 内容: 新建 \subsection{Limitations and Scope Boundaries}
% 代码: PART4_TextBlocks_Summary.tex, line 40-70
% 包含: 5个段落 - Buffer, Replication, Smoothing, FTE Scaling, Demand
%
% 位置19: Section 5.7 或 Chapter 6 - 建议/结论
% ─────────────────────────────────────
% 文件: docs/doc/5.txt, 建议章节或结论
% 内容: 新建 \subsection{Operational Recommendations}
% 代码: PART4_TextBlocks_Summary.tex, line 75-110
% 包含: 5个建议 - G2优先, 提前入库, 平滑到达, FTE调整, 监控

%%%%%%%%%%%%%%%%%%%%%%%%%%%%%%%%%%%%%%%%%%%%%%%%%%%%%%%%%%%%%%%%%%%%%%%%%%%%%%%
% ⚙️ LaTeX配置要求
%%%%%%%%%%%%%%%%%%%%%%%%%%%%%%%%%%%%%%%%%%%%%%%%%%%%%%%%%%%%%%%%%%%%%%%%%%%%%%%

% 必需包 (添加到preamble):
% \usepackage{tikz}
% \usepackage{subcaption}
% \usetikzlibrary{shapes,arrows,positioning}

% 可选 - 图片路径配置:
% \graphicspath{{../outputs/figures/}}

% 或在每个\includegraphics中使用相对路径:
% \includegraphics[width=...]{../outputs/figures/1_sla_compliance_rate.png}

%%%%%%%%%%%%%%%%%%%%%%%%%%%%%%%%%%%%%%%%%%%%%%%%%%%%%%%%%%%%%%%%%%%%%%%%%%%%%%%
% 📊 关键数据速查 - 用于验证插入的数值
%%%%%%%%%%%%%%%%%%%%%%%%%%%%%%%%%%%%%%%%%%%%%%%%%%%%%%%%%%%%%%%%%%%%%%%%%%%%%%%

% SLA表现 (所有场景):
% ┌────────────┬─────────┬─────────┬─────────┐
% │ Scenario   │ Overall │ G2      │ ROW     │
% ├────────────┼─────────┼─────────┼─────────┤
% │ Baseline   │ 91.96%  │ 89.97%  │ 100%    │
% │ Scenario 1 │ 91.39%  │ 89.27%  │ 100%    │
% │ Scenario 2 │ 89.97%  │ 87.62%  │ 100%    │
% │ Scenario 3 │ 91.06%  │ 88.81%  │ 100%    │
% └────────────┴─────────┴─────────┴─────────┘

% 等待时间 (小时):
% ┌────────────┬──────────┬──────────┬──────────┐
% │ Scenario   │ Avg      │ Max      │ P95      │
% ├────────────┼──────────┼──────────┼──────────┤
% │ Baseline   │ 0.71±0.02│ 7.02±0.32│ 3.40±0.05│
% │ Scenario 3 │ 0.86±0.03│ 9.58±0.14│ 3.58±0.24│
% │ 变化       │ +21%     │ +36%     │ +5%      │
% └────────────┴──────────┴──────────┴──────────┘

% 吞吐量 (pallets):
% ┌───────────────┬──────────┬────────────┬─────────┐
% │ Flow Type     │ Baseline │ Scenario 3 │ Change  │
% ├───────────────┼──────────┼────────────┼─────────┤
% │ Inbound       │ 116,360  │ 113,743    │ -2%     │
% │ Outbound      │ 131,015  │ 105,950    │ -19%    │
% │ FG Outbound   │ 73,227   │ 62,175     │ -15%    │
% │ R&P Outbound  │ 57,787   │ 43,775     │ -24%    │
% └───────────────┴──────────┴────────────┴─────────┘

% 利用率 (%):
% ┌────────────┬──────────┬────────────┬─────────┐
% │ Type       │ Baseline │ Scenario 3 │ Change  │
% ├────────────┼──────────┼────────────┼─────────┤
% │ Loading    │ 23.60    │ 17.92      │ -24%    │
% │ Reception  │ 31.63    │ 28.64      │ -9%     │
% │ FG Dock    │ 34.36    │ 27.88      │ -19%    │
% │ R&P Dock   │ 20.87    │ 18.67      │ -11%    │
% └────────────┴──────────┴────────────┴─────────┘

%%%%%%%%%%%%%%%%%%%%%%%%%%%%%%%%%%%%%%%%%%%%%%%%%%%%%%%%%%%%%%%%%%%%%%%%%%%%%%%
% 🚀 使用建议
%%%%%%%%%%%%%%%%%%%%%%%%%%%%%%%%%%%%%%%%%%%%%%%%%%%%%%%%%%%%%%%%%%%%%%%%%%%%%%%

% 推荐插入顺序 (从最重要到补充):
%
% 阶段1 - 核心必需 (3-4小时工作量):
%   1. 总体架构图 (ARCHITECTURE_DIAGRAMS)
%   2. Inbound & Outbound架构图 (ARCHITECTURE_DIAGRAMS)
%   3. 表5.1 主对比表 (PART2)
%   4. 图1, 1b SLA图 (PART3A)
%   5. 图5b 小时剖析 (PART3C)
%   6. 表4.1 FTE配置 (PART1)
%
% 阶段2 - 强化支撑 (2小时工作量):
%   7. 图2 等待时间 (PART3A)
%   8. 图4, 4c 吞吐量 (PART3B)
%   9. 图5 利用率 (PART3C)
%
% 阶段3 - 完整补充 (2小时工作量):
%  10. 其余图表 (PART3A/3B)
%  11. 其余表格 (PART1/PART2)
%  12. 文字段落 (PART4)
%  13. 验证列表 (PART1)

%%%%%%%%%%%%%%%%%%%%%%%%%%%%%%%%%%%%%%%%%%%%%%%%%%%%%%%%%%%%%%%%%%%%%%%%%%%%%%%
% 📞 文件位置
%%%%%%%%%%%%%%%%%%%%%%%%%%%%%%%%%%%%%%%%%%%%%%%%%%%%%%%%%%%%%%%%%%%%%%%%%%%%%%%

% LaTeX代码文件 (docs/目录下):
%   - PART1_Chapter4_Tables_Figures.tex
%   - ARCHITECTURE_DIAGRAMS.tex ⭐⭐⭐ 新增完整架构图集
%   - PART2_Chapter5_Tables.tex
%   - PART3A_Chapter5_Figures_SLA_Wait.tex
%   - PART3B_Chapter5_Figures_Throughput.tex
%   - PART3C_Chapter5_Figures_Utilization.tex
%   - PART4_TextBlocks_Summary.tex
%   - THIS FILE: MASTER_INDEX.tex

% 操作指导文件:
%   - docs/INSERTION_INSTRUCTIONS.md (中文说明)

% 数据文件:
%   - outputs/results/simulation_results_comparison.xlsx
%   - outputs/results/report_data.json

% 图片文件 (outputs/figures/目录下):
%   - 13个PNG文件 (1, 1b, 2, 3, 4, 4b, 4c, 4d, 5, 5b×4)

%%%%%%%%%%%%%%%%%%%%%%%%%%%%%%%%%%%%%%%%%%%%%%%%%%%%%%%%%%%%%%%%%%%%%%%%%%%%%%%
% ✅ 完成检查清单
%%%%%%%%%%%%%%%%%%%%%%%%%%%%%%%%%%%%%%%%%%%%%%%%%%%%%%%%%%%%%%%%%%%%%%%%%%%%%%%
7项 - 新增3个架构图):
% [ ] 表4.1 FTE配置
% [ ] 表4.2 到达随机性
% [ ] 总体架构图 (Entity/Process/Resource/Constraint) ⭐⭐⭐
% [ ] Inbound流程架构图 ⭐⭐⭐
% [ ] Outbound流程架构图 ⭐⭐⭐
% [ ] TikZ简化架构图 (PART1, 可选替代)随机性
% [ ] TikZ架构图
% [ ] 验证列表

% 第5章表格 (2项):
% [ ] 表5.1 主对比表 ⭐⭐⭐
% [ ] 表5.2 区域分解

% 第5章图片 (13项):
% [ ] 图1 SLA Overall ⭐⭐⭐
% [ ] 图1b SLA Region ⭐⭐⭐
% [ ] 图2 Avg Wait ⭐⭐
% [ ] 图3 Backlog (可选)
% [ ] 图4 Throughput Pallets ⭐⭐
% [ ] 图4b FG Region Pallets ⭐
% [ ] 图4c Orders vs Pallets ⭐⭐
% [ ] 图4d FG Region Orders ⭐
% [ ] 图5 Dock Utilization ⭐⭐
% [ ] 图5b Hourly Profiles (4图) ⭐⭐⭐

% 文字段落 (3项):
% [ ] 结果概览
% [ ] 局限性讨论
% [ ] 运营建议

%%%%%%%%%%%%%%%%%%%%%%%%%%%%%%%%%%%%%%%%%%%%%%%%%%%%%%%%%%%%%%%%%%%%%%%%%%%%%%%
% END OF MASTER INDEX
%%%%%%%%%%%%%%%%%%%%%%%%%%%%%%%%%%%%%%%%%%%%%%%%%%%%%%%%%%%%%%%%%%%%%%%%%%%%%%%
