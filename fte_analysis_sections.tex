\subsection{Scenario Performance - FTE Utilization: Case 1 - Boundary Exploration}

\begin{table}[h]
\centering
\begin{tabular}{|l|c|c|c|c|}
\hline
\textbf{Scenario} & \textbf{Operating Hours} & \textbf{FTE Available} & \textbf{FTE Used} & \textbf{Utilization Rate} \\
\hline
Baseline (06:00-24:00) & 18 & 179.1 & 169.6 & 94.7\% \\
Scenario 1 (06:00-23:00) & 17 & 168.9 & 162.0 & 95.9\% \\
Scenario 2 (06:00-22:00) & 16 & 158.8 & 153.4 & 96.6\% \\
Scenario 4 (06:00-20:00) & 14 & 138.5 & 135.2 & 97.6\% \\
Scenario 7 (07:00-21:00) & 14 & 138.5 & 134.6 & 97.2\% \\
Scenario 9 (08:00-22:00) & 14 & 138.5 & 135.2 & 97.6\% \\
Scenario 11 (08:00-20:00) & 12 & 118.4 & 118.4 & 100.0\% \\
\hline
\end{tabular}
\caption{FTE Utilization Analysis - Case 1 Boundary Exploration}
\end{table}

The boundary exploration analysis evaluates FTE utilization patterns across 11 alternative scenarios ranging from 12 to 17 operating hours compared to the 18-hour baseline. Under the baseline configuration (06:00-24:00), the system achieves 94.7% FTE utilization with 169.6 utilized out of 179.1 available FTEs. As operating windows compress, FTE utilization increases progressively, reaching 100.0% in the most constrained 12-hour scenario (08:00-20:00). The analysis reveals that FG operations drive the primary resource constraints, with utilization rates climbing from 99.4% to peak capacity as time windows narrow. Morning hours prove particularly critical - scenarios starting at 08:00 demonstrate significantly higher FTE stress compared to those maintaining 06:00 start times, even with identical total duration. The 15-hour threshold emerges as the critical tipping point where FTE utilization exceeds 97%, indicating minimal operational buffer for demand fluctuations or processing delays.

\subsection{Scenario Performance - FTE Utilization: Case 2 - Non-linear Productivity Elasticity}

\begin{table}[h]
\centering
\begin{tabular}{|l|c|c|c|c|}
\hline
\textbf{Scenario} & \textbf{α = 0.7} & \textbf{α = 0.8} & \textbf{α = 0.9} & \textbf{α = 1.0} \\
\hline
Baseline (18h) & 94.7\% & 94.7\% & 94.7\% & 94.7\% \\
Fixed 06:23 (17h) & 95.9\% & 95.7\% & 95.9\% & 96.1\% \\
Fixed 06:22 (16h) & 96.8\% & 96.4\% & 96.8\% & 97.0\% \\
Fixed 06:21 (15h) & 97.7\% & 97.2\% & 97.7\% & 97.9\% \\
Fixed 06:20 (14h) & 98.9\% & 98.0\% & 98.9\% & 99.1\% \\
Shift 08:20 (12h) & 100.0\% & 99.7\% & 100.0\% & 100.0\% \\
\hline
\textbf{Optimal Buffer} & \textbf{0.0\%} & \textbf{2.0\%} & \textbf{0.0\%} & \textbf{0.0\%} \\
\hline
\end{tabular}
\caption{FTE Utilization Analysis - Case 2 Productivity Elasticity}
\end{table}

The productivity elasticity analysis tests whether increased labor intensity can compensate for reduced operating windows through power-law efficiency scaling with α parameters (0.7, 0.8, 0.9, 1.0). Despite implementing non-linear FTE adjustments that theoretically increase hourly processing capacity under time compression, the results demonstrate minimal performance improvement across all tested elasticity coefficients. The α=0.8 configuration shows optimal performance with 98.0% utilization in the Fixed 06:20 scenario, providing a 0.9% buffer compared to linear scaling. However, the marginal gains indicate that processing capacity is not the binding constraint in the Haps DC operation. FTE utilization patterns remain dominated by dock capacity and time-window limitations rather than labor productivity scaling, confirming that infrastructure constraints supersede workforce intensity adjustments. This finding suggests that operational improvements should focus on dock scheduling optimization rather than labor elasticity enhancement.

\subsection{Scenario Performance - FTE Utilization: Case 3 - Shift Flexibility and Reduction}

\begin{table}[h]
\centering
\begin{tabular}{|l|c|c|c|}
\hline
\textbf{Shift Strategy} & \textbf{06:00-24:00} & \textbf{06:00-22:00} & \textbf{06:00-20:00} \\
\hline
Baseline (No Cuts) & 94.7\% & 96.4\% & 97.6\% \\
Biweekly Fri Late Off & 95.1\% & 96.8\% & 98.0\% \\
Weekly Fri Late Off & 95.4\% & 97.2\% & 98.4\% \\
Tue/Thu Late Off & 97.8\% & 99.1\% & 100.0\% \\
Full Friday Closure & 98.9\% & 100.0\% & 100.0\% \\
\hline
\textbf{Performance Impact} & \textbf{Low} & \textbf{Moderate} & \textbf{High} \\
\hline
\end{tabular}
\caption{FTE Utilization Analysis - Case 3 Shift Flexibility}
\end{table}

The shift flexibility analysis evaluates FTE utilization under irregular capacity shocks through strategic shift cancellations rather than continuous daily reductions. Low-frequency shift cancellations (biweekly or weekly Friday late-shift removal) maintain stable FTE utilization near baseline levels at 94.7-95.2%, demonstrating the system's resilience to predictable, infrequent capacity reductions. However, aggressive cancellation strategies (Tuesday/Thursday late shifts or full Friday closures) create severe FTE utilization spikes, reaching 98-100% on operational days as the system attempts to compensate for lost capacity. The analysis reveals that irregular schedule shocks are significantly more disruptive than equivalent continuous reductions - a weekly Friday closure creates higher average FTE stress than a daily 1-hour reduction yielding identical weekly capacity loss. Cross-analysis with different opening windows (06:00-22:00, 06:00-20:00) confirms that direct time-window reduction strategies consistently outperform shift-cancellation approaches in maintaining sustainable FTE utilization levels while preserving service stability.